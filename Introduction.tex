\ihead{Einleitung}
\section{Einleitung}
Jeder vierte Todesfall eines Menschen in der Europäischen Union wird durch eine Krebserkrankung verursacht. Die Organisation Eurostat berichtet, dass Krebs im Jahr 2013 für 26\% (in Zahlen 1.300.000) aller Todesfälle verantwortlich ist. Wenn man dies noch weiter aufschlüsselt, sterben 12\% (in Zahlen 153.100) aller krebsbedingten Todesfälle an Dickdarmkrebs. Somit ist der Dickdarmkrebs, die zweithäufigste Todesursache bei den Krebspatienten in der Europäischen Union \cite{Eurostat2016}. Die American Cancer Society (weiters als ACS bezeichnet) beschreibt in ihrem Bericht aus dem Jahr 2015, dass Dickdarmkrebs für 693.900 Tote weltwelt verantwortlich ist.  Außerdem schreibt die ACS, dass in höher entwickelten Staaten öfters Dickdarmkrebs diagnostiziert wird. Für Frauen ist es die zweithäufigste und für Männer die dritthäufigste Krebsdiagnose \cite{Torre2015}. Dies soll zum Anlass genommen werden, um über eine frühzeitige und korrekte Diagnose von Dickdarmkrebs mit Hilfe von „State oft the Art“ Technologien zu diskutieren. Die Medizinische Bildverarbeitung bietet Technologien um Ärztinnen und Ärzten einen besseren Einblick in den menschlichen Körper zu ermöglichen. Diese Methoden können Patientinnen und Patienten mit einer höheren Auflösung, Qualität und Präzision darstellen. Weiteres kann die Technik dem Medizinern bei der Diagnostik und der Therapie unterstützen und somit mithelfen dem Patienten frühestmöglich, die bestmögliche Behandlung zukommen zu lassen \cite{Handels2009}. Der Kampf gegen den Krebs ist in den letzten Jahren durch Forschungsteams im Bereich der Künstlichen Intelligenz aufgenommen worden. Durch Technologien wie Neuronale Netze, Machine Learning und Deep Learning sind in den letzten Jahren weitere Möglichkeiten entstanden, mit welchen man Maschinen das Sehen ermöglichen kann. Dieser Bereich wird in der Forschung oft als „Computer Vision“ bezeichnet. Die ersten Studien in diesem Bereich fokussierten sich vor allem auf die Featureextrahierung aus den Bilddaten, diese Features mussten von einem Experten erfasst und evaluiert werden. Der Prozess stellte sich allerdings als Limitierung für bessere Ergebnisse heraus, weil das Resultat von der Arbeit des Experten abhing. Die Lösung für dieses Problem ist Representation Learning, mit dieser Methode können Repräsentationen/Features aus den Daten maschinell gelernt werden. Die Deep Learning Algorithmen sind eine Art von Representation Learning, welche Repräsentation in den Bilddaten finden und lernen können. Der Vorteil von Deep Learning ist daher, dass sehr komplexe hierarchische Repräsentationen generiert werden können, welche in weiterer Folge zum Trainieren eines Neuronalen Netzes verwendet werden \cite{Hu2018}. 

In dieser Masterarbeit soll somit versucht werden Darmkrebs in 3D CT Scans mittels Deep Learning Algorithmen zu klassifizieren und somit eine Möglichkeit zur frühzeitigen Erkennung von Krebserkrankungen zu schaffen.
